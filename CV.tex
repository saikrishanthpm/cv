
\documentclass[11pt]{article}
\usepackage{geometry}
\usepackage{enumitem}
\usepackage{multicol}
\pagestyle{empty}
\geometry{margin=0.5in}
\setlength\parindent{9pt}
\setlist{nosep}
\newcommand{\name}[1]{\begin{center}\section*{\huge #1}\end{center}}
\newcommand{\topinfo}[1]{\begin{center}\vspace{-0.2cm}#1\vspace{-0.2cm}\end{center}}
\newcommand{\resumesection}[1]{\vspace{-0.2cm}\section*{#1}\vspace{-0.2cm}\hrule\vspace{0.2cm}}



\begin{document}
\name{Sai Krishanth PM}
\topinfo{822 E Lee St, Apt 21, Tucson, AZ, 85719}
\topinfo{saikrishanth@arizona.edu}
\topinfo{520-528-6436}

\resumesection{Education}
\textbf{University of Arizona}, College of Science \hfill August 2018--May 2022
\\ B.S. Physics and Astronomy with a minor in Mathematics 

\resumesection{Relavent Coursework}
\begin{multicols}{3}
\noindent Nuclear \& Particle Physics \\
General Relativity (Graduate) \\
Quantum Mechanics I \& II\\
Statistical Mechanics \\
Electromagnetism \\
Computational Physics \\
Optics \\
Cosmology and Extragalactic \\
Astronomy \\
Stellar Structure and Evolution \\
Radiative Processes\\
Astrophysical Dynamics\\
Mathematical Proofs\\
Linear Algebra\\
Applied Mathematics\\
Fourier Analysis\\
\end{multicols}

\resumesection{Research Experience}
\textbf{Characterizing HI distributions in nearby massive spiral galaxies}  \hfill September 2018--May 2022
\\\textbf{Advisor:} Dr. Alyson Ford, Steward Observatory
\begin{itemize}
	\item Reducing Green Bank Telescope (GBT) spectroscopic data of neutral hydrogen (HI) for low redshift spiral galaxies 
	\item Calculating column density and error boundaries from generated FITS files
	\item Co-writing proposal for an expanded survey based on the results of reduced data and observing targets at GBT for said survey  
	\item Reducing HI data from quiescent elliptical galaxies and creating deep maps to measure cool gas content and distribution. 
\end{itemize}
\textbf{Pipeline development for the NEID spectrograph} \hfill January 2020-August 2021
\\\textbf{Advisor:} Dr. Chad Bender, Steward Observatory
\begin{itemize}
	\item Manually checking for errors in lightcurves obtained by the NEID spectrograph
	\item Debugging the NEID control software 
	\item Writing and implementing error correcting code in the NEID data collection pipeline
	\item Designing and building shutter control boxes for the NEID and HPF spectrographs.
\end{itemize}
\textbf{Determining the dominant source of uncertainty in FoM calculations} \hfill July 2020-Present
\\ \textbf{Advisor:} Dr. Tim Eifler, Steward Observatory
\begin{itemize}
	\item Running simulated likelihood analysis of Dark Energy Survey Year 3 data (DES Y3)
	\item Quantifying the dominant source of uncertainty by performing figure of merit (FoM) calculations.
	\item Using Self organizing maps (SOMs) to generate photometric redshift probability distribution functions (PDFs). 
\end{itemize}
\textbf{Internship at Paramium Technologies}  \hfill May 2021-August 2021
\\\textbf{Advisor:} Dr. Justin Hyatt
\begin{itemize}
	\item Building a test bench to generate a hysteresis curve of the stepper motor response on an adjustable radio dish mold
	\item Designing and 3D printing components for use in the test bench
	\item Quantifying test results by writing a technical report and generating a GIF of the mold for qualitative analysis
\end{itemize}
\textbf{Testing Universal Relations in Neutron Stars}  \hfill September 2021--Present
\\\textbf{Advisor:} Dr. Vasileios Paschalidis, Steward Observatory
\begin{itemize}
	\item Quantifying violations in universal relations using millions of equations of states of neutron stars 
	\item Implementing momentum of inertia computations for equations of state in the slow rotation limit
\end{itemize}

\resumesection{Proposals and Publications}

\begin{itemize}
	\item \textit{A Survey of Extended HI Disks Around Nearby Galaxies}, Alyson Ford, Joel Bregman, Edmund Hodges-Kluck, Jeremy Bailin, Michael Hardegree-Ullman, \textbf{Sai Krishanth Pulikesi Mannan}. Observing during the Spring 2022 semester at GBT. 
	\item \textit{Accelerating cosmological inference with Gaussian processes and neural
	networks - an application to LSST Y1 weak lensing and galaxy clustering}, Supranta S. Boruah, Tim Eifler, Vivian Miranda, \textbf{Sai Krishanth P.M.}. Submitted to MNRAS. 
\end{itemize}

\resumesection{Technical Skills}

\textbf{Programming Languages}: Python, C++, IDL, IRAF 
\\
\textbf{Software}: MATLAB, GBTIDL (Custom version of IDL for GBT), \LaTeX, SLURM, AstroImageJ, DS9, Solidworks, Adobe Inventor
\\
\textbf{Other}: Certified observer at GBT, Soldering, Circuit Design

\resumesection{Outreach}
\textbf{Vice President, Astronomy Club} \hfill December 2019 - September 2021
\begin{itemize}
	\item Student led organization to promote outreach and education in Astronomy. 
	\item Responsibilities: 
	\begin{itemize}
		\item Facilitated the Astrophotography program in the club.
		\item Organized and planned field trips.
		\item Acted as liaison between officers and club members
		\item Lead the diversity and inclusion initiative.
		\item Organized a graduate student panel to discuss diversity and equity in STEM.
	\end{itemize}	       
\end{itemize} 
\textbf{TIMESTEP Leader, Steward Observatory} \hfill August 2020 - January 2021
\begin{itemize}
	\item Organization to promote minority student engagement and retention in STEM. 
	\item Responsibilities: 
	\begin{itemize}
		\item Participated in a discussion panel about undergraduate research. 
		\item Aided in organizing other panels and proposed future meeting ideas. 
	\end{itemize}    
\end{itemize} 

\resumesection{Talks and Presentations}
\begin{itemize}
	\item Astronomy club internal colloquium 
	\item Astronomy news of the week presentations for astronomy club 
	\item TIMESTEP summer internship summary presentation
	\item Splendido retirement community astronomy research presentation
\end{itemize}

\end{document}