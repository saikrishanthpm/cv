\documentclass[11pt]{article}
\usepackage{geometry}
\usepackage{enumitem}
\usepackage{multicol}
\usepackage{hyperref}
\pagestyle{empty}
\geometry{margin=0.5in}
\setlength\parindent{9pt}
\setlist{nosep}
\newcommand{\name}[1]{\begin{center}\section*{\huge #1}\end{center}}
\newcommand{\topinfo}[1]{\begin{center}\vspace{-0.2cm}#1\vspace{-0.2cm}\end{center}}
\newcommand{\resumesection}[1]{\vspace{-0.2cm}\section*{#1}\vspace{-0.2cm}\hrule\vspace{0.2cm}}



\begin{document}
\name{Sai Krishanth P.M.}
\topinfo{N435, Steward Observatory, 933 N Cherry Ave, Tucson, AZ, 85719}
\topinfo{saikrishanth@arizona.edu}
\topinfo{520-626-6958}

\resumesection{Education}
\textbf{University of Arizona}, College of Science \hfill August 2018-May 2022
\\ B.S. Astronomy and Physics with a minor in Mathematics 

\resumesection{Research Experience}
\textbf{Research Staff at The University of Arizona}  \hfill September 2022-Present
\\\textbf{Advisor:} Dr. Ewan Douglas, Steward Observatory
\begin{itemize}
	\item Wrote an NMF based post-processing pipeline for use with high contrast imaging data from ground and space based observatories.
    \item Made the NMF implementation faster by rewriting code to run on GPUs.
    \item Built a website for the Center for Astronomical adaptive optics.
\end{itemize}
\textbf{Testing Universal Relations in Neutron Stars}  \hfill September 2021--Present
\\\textbf{Advisor:} Dr. Vasileios Paschalidis, Steward Observatory
\begin{itemize}
	\item Quantified violations in universal relations using millions of equations of states of neutron stars. 
	\item Implemented momentum of inertia computations for equations of state in the slow rotation limit.
\end{itemize}
\textbf{Determining the dominant source of uncertainty in FoM calculations} \hfill July 2020-May 2022
\\ \textbf{Advisor:} Dr. Tim Eifler, Steward Observatory
\begin{itemize}
	\item Ran simulated likelihood analysis of DES Y3 and LSST Y1 data.
	\item Quantified the dominant source of uncertainty by performing figure of merit (FoM) calculations.
	\item Used self organizing maps (SOMs) to generate photometric redshift probability distribution functions (PDFs). 
\end{itemize}
\textbf{Characterizing HI distributions in nearby massive spiral galaxies}  \hfill September 2018--May 2022
\\\textbf{Advisor:} Dr. Alyson Ford, Steward Observatory
\begin{itemize}
	\item Reduced Green Bank Telescope (GBT) spectroscopic data of neutral hydrogen (HI) for low redshift spiral galaxies. 
	\item Calculated column density and error boundaries from generated FITS files.
	\item Co-wrote proposal for an expanded survey based on the results of reduced data and observed targets at GBT.  
	\item Reduced HI data from quiescent elliptical galaxies and created deep maps to measure cool gas content and distribution. 
\end{itemize}
\textbf{Pipeline development for the NEID spectrograph} \hfill January 2020-August 2021
\\\textbf{Advisor:} Dr. Chad Bender, Steward Observatory
\begin{itemize}
	\item Manually checked for errors in lightcurves obtained by the NEID spectrograph.
	\item Debugged the NEID control software. 
	\item Wrote and implemented error-correcting code in the NEID data collection pipeline.
	\item Designed and built shutter control boxes for the NEID and HPF spectrographs.
\end{itemize}
\textbf{Summer internship at Paramium Technologies}  \hfill May 2021-August 2021
\\\textbf{Advisor:} Dr. Justin Hyatt
\begin{itemize}
	\item Built a test bench to generate a hysteresis curve of the stepper motor response on an adjustable radio dish mold test bench.
	\item Designed and 3D printed components for use in the test bench.
	\item Quantified test results by writing a technical report and generating a GIF of the mold for qualitative analysis.
\end{itemize}

\resumesection{Publications}

\begin{enumerate} 
    \item \textit{Deepest limits on scattered light emission from the Epsilon Eridani inner debris disk with HST/STIS} \\
    \textbf{Sai Krishanth P.M.}, Ewan S. Douglas, Ramya Anche, Justin Hom, John H. Debes, Kerri Cahoy, Hannah Jang-Condell, 
	Isabel Rebollido, Bin B. Ren, Christopher C. Stark, Robert Thompson, Yinzi Xin, submitted to the Astronomical Journal (2024).
    %\item \textit{A comparison of post-processing methods for exoplanet coronagraphy}\\
    %\textbf{Sai Krishanth P.M.} et al., (in prep. for APJ Letters).
    %\item \textit{An RDI library for JWST coronagraphy}\\
    %\textbf{Sai Krishanth P.M.} et al., submitted to APJ Letters.
    \item \textit{NMF-based GPU accelerated coronagraphy pipeline}\\ \textbf{Sai Krishanth P.M.}, Ewan S. Douglas, Justin Hom, Ramya M. Anche, John Debes, Isabel Rebollido, Bin B. Ren, Proceedings of the SPIE, Volume 12680, id. 1268021 12 pp. (2023), \url{https://doi.org/10.1117/12.2677739}.
    \item \textit{A temperate super-Jupiter imaged with JWST in the mid-infrared}\\ E. C. Matthews, A. Carter, P. Pathak, C. Morley, M. W. Phillips, \textbf{S. Krishanth P.M.}, F. Feng, M. Bonse, L. Boogaard, J. Burt, I. J. M. Crossfield, E. S. Douglas, Th. Henning, J. Hom, C.-L. Ko, A.-M. Lagrange, D. Petit dit de la Roche, F. Philipot, Nature (2024), \url{https://doi.org/10.1038/s41586-024-07837-8}
    \item \textit{Accelerating cosmological inference with Gaussian processes and neural networks - an application to LSST Y1 weak lensing and galaxy clustering}\\
    Supranta S. Boruah, Tim Eifler, Vivian Miranda, \textbf{P M Sai Krishanth}, Monthly Notices of the Royal Astronomical Society, Volume 518, Issue 4, February 2023, Pages 4818–4831, \url{https://doi.org/10.1093/mnras/stac3417}. 
    \item \textit{Approaches to lowering the cost of large space telescopes}\\
    Ewan S. Douglas et al. credited as \textbf{Sai Krishanth P.M.}, Proceedings of the SPIE, Volume 12677, id. 126770D 20 pp. (2023), \url{https://doi.org/10.1117/12.2677843}.
\end{enumerate}

\resumesection{Proposals}
\begin{enumerate} 
    \item \textit{A Survey of Extended HI Disks Around Nearby Galaxies}, Alyson Ford, Joel Bregman, Edmund Hodges-Kluck, Jeremy Bailin, Michael Hardegree-Ullman, \textbf{Sai Krishanth Pulikesi Mannan}, Observed during the Spring 2022 semester at GBT, \url{https://dss.gb.nrao.edu/project/GBT22A-287/public}.
\end{enumerate}
\resumesection{Technical Skills}

\textbf{Programming Languages}: Python, C++, Fortran, IDL, HTML, CSS, JavaScript, \LaTeX 
\\
\textbf{Software}: MATLAB, GBTIDL (Custom version of IDL for GBT), IRAF, SLURM, AstroImageJ, DS9, Solidworks, Adobe Inventor
\\
\textbf{Other}: Certified observer at GBT, Soldering, Circuit Design, Contributions to open source software (Poppy, \url{https://github.com/spacetelescope/poppy})

\resumesection{Outreach}
\textbf{Vice President, Astronomy Club} \hfill December 2019 - September 2021
\begin{itemize}
	\item Student-led organization to promote outreach and education in Astronomy. 
	\item Responsibilities: 
	\begin{itemize}
		\item Facilitated the Astrophotography program in the club.
		\item Organized and planned field trips.
		\item Acted as liaison between officers and club members
		\item Led the diversity and inclusion initiative.
		\item Organized a graduate student panel to discuss diversity and equity in STEM.
	\end{itemize}	       
\end{itemize} 
\textbf{TIMESTEP Leader, Steward Observatory} \hfill August 2020 - January 2021
\begin{itemize}
	\item Organization to promote minority student engagement and retention in STEM. 
	\item Responsibilities: 
	\begin{itemize}
		\item Participated in a discussion panel about undergraduate research. 
		\item Aided in organizing other panels and proposed future meeting ideas. 
	\end{itemize}    
\end{itemize} 

\resumesection{Talks and Presentations}
\begin{itemize}
	\item Colloquium, Astronomy club, Depart of Astronomy, Tucson, AZ, USA.  
    \item Poster, SPIE O+P 2023, San Diego, CA, USA.
    \item Colloquium, High Contrast Images of Exoplanets (HICE) talk series, Steward Observatory, Tucson, AZ, USA.
    \item Contributed talk, Dust Devils: Debris disks in the Sonoran desert, Tucson, AZ, USA.
\end{itemize}

\end{document}
